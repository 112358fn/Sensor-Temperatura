% Skype network topology
% Author Fiandrino Claudio 2011
% http://claudiofiandrino.altervista.org/

%Modified by Alvaro Alonso
% September 2014

\begin{figure}[H]
  \begin{center} 
    \resizebox{2.5in}{!}{
\begin{tikzpicture}[auto, thick]
  % Place BeagleBone
   \node[beaglebone] (a) at (0,3) {\small{BeagleBone}};
  % Place super peers and connect them
  \foreach \place/\name in {{(-2,2)/f}, {(2,0)/b}, {(2,2)/c}, {(0,-1)/d},
           {(-2,0)/e}}
    \node[pic18f] (\name) at \place {};
  \foreach \source/\dest in {a/b, a/c, a/d, a/e, a/f}
    \path (\source) edge (\dest);
   %
   % Place normal peers
  \foreach \pos/\i in {above left of/1, left of/2, below left of/3}
    \node[sensor, \pos = e] (e\i) {};
   \foreach \speer/\peer in {e/e1,e/e2,e/e3}
    \path (\speer) edge (\peer);
   %
   \foreach \pos/\i in {above left of/1, left of/2, below left of/3}
    \node[sensor, \pos = f] (f\i) {};
   \foreach \speer/\peer in {f/f1,f/f2,f/f3}
    \path (\speer) edge (\peer);
    %
   \foreach \pos/\i in {above right of/1, right of/2, below right of/3}
    \node[sensor, \pos =b ] (b\i) {};
   \foreach \speer/\peer in {b/b1,b/b2,b/b3}
   \path (\speer) edge (\peer);
   %
   \foreach \pos/\i in {below right of/1, below of/2, below left of/3}
   \node[sensor, \pos =d ] (d\i) {};
   \foreach \speer/\peer in {d/d1,d/d2,d/d3}
   \path (\speer) edge (\peer);
   %
   \foreach \pos/\i in {right of/1, above of/2}
   \node[sensor, \pos =c ] (c\i) {};
   \foreach \speer/\peer in {c/c1,c/c2}
   \path (\speer) edge (\peer);
   %%%%%%%%
   % Legends
   \node[sensor] at (5,0) {\small{Sensores}};
   \node[pic18f] at (5,2) {\small{PIC18F}};
   \node[legend_general] at (0,4) {\small{\textsc{Topología general}}};
\end{tikzpicture}
    }
  \end{center}
  \caption{Topologia general del sistema}
  \label{fig:topologia}
\end{figure}