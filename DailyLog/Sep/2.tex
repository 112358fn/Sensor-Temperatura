\labday{Lunes, 13 Septiembre 2014}
\experiment{Selección de un microcontrolador}
Si bien la semana pasada se había optado por utilizar un microcontrolador de la familia $18Fxx$ 
resulta de interés el disminuir los costos del sistema por esta razón se contempla la posibilidad
de usar algunos de los siguientes $PICs$

\begin{table}[H]
    \begin{tabular}{l l l}
    \toprule
    \textbf{Patas} & \textbf{Antiguos} & \textbf{Nuevos} \\
    \toprule
    28 & 16F873 & 16F883\\
    28 & 16F876 & 16F886\\
    40 & 16F874 & 16F884\\
    40 & 16F877 & 16F887\\
    \bottomrule
    \end{tabular}
  \caption{Opciones de $PIC 16F$}
  \label{tab:comparacionPIC}
\end{table}

Dentro de la tabla\ref{tab:comparacionPIC} el $PIC16F884$ se encuentra disponible en el mercado local.
Esto lo posiciona por sobre los otros a la hora de la selección. Se comienza el análisis con este 
último PIC.

La información correspondiente al $PIC16F884$ se puede obtener de\\
\url{http://www.microchip.com/wwwproducts/Devices.aspx?dDocName=en026564}

\experiment{Seleccion de un entorno de desarrollo}
Buscamos encontrar un entorno de desarrollo que sea libre para poder ser usado sin restricciones.
Se plantean así las siguientes soluciones
\begin{itemize}
 \item Netbeans
 \item Eclipse
\end{itemize}
Sobre este ultimo se cuenta con experiencia en el desarrollo de aplicaciones embebidas y en C. Por tal
razon es que resulta mas interesante poder encontrar una solucion en este entorno.\\
\url{http://kr3l.wordpress.com/2008/11/02/using-eclipse-for-pic-development/}