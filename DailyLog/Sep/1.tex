\labday{Lunes, 1 Septiembre 2014}
\experiment{Sistema de gestión de versiones: Git}

Puesta en marcha del sistema de gestión de versiones Git: \url{http://git-scm.com/}.
Ver también: \url{http://en.wikipedia.org/wiki/Git_%28software%29}\\
Para poder trabajar de manera colaborativa se utiliza el servidor remoto
de github \url{https://github.com/}. Este servicio es gratis para repositorios 
que permanezcan abiertos a todo el publico. También, presenta el beneficio de 
plantear un marco para poder compartir desarrollos y códigos de manera libre.
En particular en este desarrollo se opto por una licencia GPL v2.

El esquema del repositorio es el siguiente:
\dirtree{%			
.1 Sensor-Temperatura/ \DTcomment{Directorio Raiz}.
.2 DailyLog/ \ldots{} \begin{minipage}[t]{5cm}
			  Este directorio contiene el \LaTeX\ del diario 
			  de desarrollo(*.tex ){.}
		      \end{minipage}.
.3 dailyLog.tex \DTcomment{Archivo principal}.
.3 Sep.
.4 1-4.tex \DTcomment{Archivo Semanal}.
.3 Oct.
.4 1-4.tex.
.3 Nov.
.4 1-4.tex.
.2 Doc \ldots{} \begin{minipage}[t]{5cm}
			  En este directorio se encuentra toda la documentación utilizada 
			  durante el desarrollo{.}
		      \end{minipage}.
.3 Datasheets.
.3 Lista de Componentes.
.2 Design \ldots{} \begin{minipage}[t]{5cm}
			  Directorio destinado a los diseños de esquemáticos
			  y PCB{.}
		      \end{minipage}.
.3 LM35.
.4 LM35-Remote.sch \DTcomment{Esquemático}.
}


\experiment{Selección de sensor de temperatura}
Para la selección del sensor se tuvo en cuenta principalmente el material disponible.
De esta manera las posibilidades de un sensor de temperatura que se pueden encontrar 
en el mercado local se reducen a:
\begin{itemize}
 \item LM335
 \item LM35
 \item Termistores comunes (1~10K)
\end{itemize}
Las hojas de datos(Datasheets) de los componentes LM335 y LM35 se encuentran en el directorio
$Doc/$. Por otro lado los termistores son dejados como segundas opciones debido a que no son
lineales como los dos anteriores, lo que representa una mayor complejidad en el ajuste de la 
curva y una posible perdida de precisión. 
Ver: \url{http://es.wikipedia.org/wiki/Termistor#Introducci.C3.B3n}

Ambos sensores son analógicos con salidas en voltaje.
\begin{table}[H]
  \begin{tabular}{l l l}
    \toprule
    \textbf{Sensor} & \textbf{Factor de escala} & \textbf{Rango de operación} \\
    \toprule
    LM335 & $10mV/^o K$ & $-40~a~100 ^o C$\\
    LM35  & $10mV/^o C$ & $-55~a~150 ^o C$\\
    \bottomrule
  \end{tabular}
  \caption{Comparación de los sensores LM335 y LM35}
  \label{tab:compSens}
\end{table}

Como podemos observar de la Tabla \ref{tab:compSens} el sensor $LM35$ presenta beneficios 
en cuanto al rango y no es necesaria la conversión de $^o K$ a $^o C$ no requiriendo la
sustracción de una gran constante de voltaje. Por tales motivos se opta por utilizar el
$LM35$ como sensor para el proyecto.

\experiment{Preselección de PIC 18Fxx}
Se desea tener un sistema en el que un nodo central(beagleBone) se comunique con nodos esclavos
los cuales deben ser capaces de medir la temperatura de 3 a 6 puntos diferentes. La comunicacion
entre nodos es digital y es el nodo esclavo el encargado de la conversion analogica/digital
y de la transmision al nodo central.

Estas razones nos llevan a elegir un microcontrolador con los suficientes canales ADC para
poder realizar la conversion y un UART para la transmision digital de los datos. Los PIC de 
la familia $18Fxx$ presentan caracteristicas que los hacen interesantes para este proyecto.

%--------------------------------------------------
\labday{Martes, 2 Septiembre 2014}
\experiment{Topologia de un nodo esclavo}
Para cumplir con las siguientes condiciones del proyecto:
\begin{itemize}
 \item Multiples puntos de sensado
 \item Distancias variables entre los puntos y el nodo central.
 \item Numero variable de puntos.
\end{itemize}
Se opta por utilizar una topologia como la que se observa en la Fig.

De esta manera el nodo esclavo es capaz de sensar un numero $x$ de puntos, cercanos al mismo
y transmitirlos de manera digital al nodo central, de manera que la distancia no sea un factor 
tan influyente. Ademas, este sistema permite la utilizacion de un bus de comunicacion, haciendo 
que el mismo pueda ser escalable, adaptandose facilmente a un mayor numero de
puntos de medida.

\experiment{Sensor LM35: Esquema de Conexión}
