%%%%%%%%%%%%%%%%%%%%%%%%%%%%%%%%%%%%%%%%%
% Daily Laboratory Book
% LaTeX Template
%
% This template has been downloaded from:
% http://www.latextemplates.com
%
% Original author:
% Frank Kuster (http://www.ctan.org/tex-archive/macros/latex/contrib/labbook/)
%
% Important note:
% This template requires the labbook.cls file to be in the same directory as the
% .tex file. The labbook.cls file provides the necessary structure to create the
% lab book.
%
% The \lipsum[#] commands throughout this template generate dummy text
% to fill the template out. These commands should all be removed when 
% writing lab book content.
%
% HOW TO USE THIS TEMPLATE 
% Each day in the lab consists of three main things:
%
% 1. LABDAY: The first thing to put is the \labday{} command with a date in 
% curly brackets, this will make a new page and put the date in big letters 
% at the top.
%
% 2. EXPERIMENT: Next you need to specify what experiment(s) you are 
% working on with an \experiment{} command with the experiment shorthand 
% in the curly brackets. The experiment shorthand is defined in the 
% 'DEFINITION OF EXPERIMENTS' section below, this means you can 
% say \experiment{pcr} and the actual text written to the PDF will be what 
% you set the 'pcr' experiment to be. If the experiment is a one off, you can 
% just write it in the bracket without creating a shorthand. Note: if you don't 
% want to have an experiment, just leave this out and it won't be printed.
%
% 3. CONTENT: Following the experiment is the content, i.e. what progress 
% you made on the experiment that day.
%
%%%%%%%%%%%%%%%%%%%%%%%%%%%%%%%%%%%%%%%%%

%----------------------------------------------------------------------------------------
%	PACKAGES AND OTHER DOCUMENT CONFIGURATIONS
%----------------------------------------------------------------------------------------

\documentclass[idxtotoc,hyperref,openany]{labbook} % 'openany' here removes the gap page between days, erase it to restore this gap; 'oneside' can also be added to remove the shift that odd pages have to the right for easier reading

\usepackage[ 
  backref=page,
  pdfpagelabels=true,
  plainpages=false,
  colorlinks=true,
  bookmarks=true,
  pdfview=FitB]{hyperref} % Required for the hyperlinks within the PDF
  
\usepackage{booktabs} % Required for the top and bottom rules in the table
\usepackage{float} % Required for specifying the exact location of a figure or table
\usepackage[pdftex]{graphicx} % Required for including images
\usepackage{lipsum} % Used for inserting dummy 'Lorem ipsum' text into the template
\usepackage[utf8]{inputenc}
\usepackage{dirtree}
\usepackage{todonotes}
\usepackage[cmex10]{amsmath}
% declare the path(s) where your graphic files are
\graphicspath{{./Sep/images/}{./Oct/images/}{./Nov/images/}}
% and their extensions so you won't have to specify these with
% every instance of \includegraphics
\DeclareGraphicsExtensions{.pdf,.jpeg,.png}

\usepackage{tikz}

\definecolor{violet}{cmyk}{0.79,0.88,0,0}
\definecolor{burntorange}{cmyk}{0,0.52,1,0}

\tikzstyle{beaglebone}=[draw, rectangle, rounded corners,red, fill=red,text=white,
                        minimum width=2.5cm, minimum height=0.8cm]
\tikzstyle{sensor}=[draw,circle,blue,fill=blue, text=white,minimum width=10pt]
\tikzstyle{pic18f}=[draw,circle,green, fill=green,text=white,minimum width=20pt]
\tikzstyle{legend_general}=[rectangle, rounded corners, thin,
                           burntorange, fill= white, draw, text=violet,
                           minimum width=2.5cm, minimum height=0.8cm]
                           

                           
\usepackage{chngcntr}
\counterwithout{figure}{labday}
\counterwithout{table}{labday}

\newcommand{\HRule}{\rule{\linewidth}{0.5mm}} % Command to make the lines in the title page
\setlength\parindent{0pt} % Removes all indentation from paragraphs

%----------------------------------------------------------------------------------------
%	DEFINITION OF EXPERIMENTS
%----------------------------------------------------------------------------------------

%\newexperiment{shorthand}{Description of the experiment}

%---------------------------------------------------------------------------------------

\begin{document}

%----------------------------------------------------------------------------------------
%	TITLE PAGE
%----------------------------------------------------------------------------------------

\frontmatter % Use Roman numerals for page numbers
\title{
\begin{center}
\HRule \\[0.4cm]
{\Huge \bfseries Diario de Desarrollo \\[0.5cm] \Large Práctica Profesional Supervisada}\\[0.4cm] % Degree
\HRule \\[1.5cm]
\end{center}
}
\author{\LARGE ALONSO, Alvaro | BADALONI, Maximiliano \\ \\ \Large 112358.fn@gmail.com | maxi.badaloni@gmail.com \\[2cm]} % Your name and email address
\date{Comienzo 1 Septiembre, 2014} % Beginning date
\maketitle

\tableofcontents

\mainmatter % Use Arabic numerals for page numbers

%----------------------------------------------------------------------------------------
%	LAB BOOK CONTENTS
%----------------------------------------------------------------------------------------
%\labday{Day, Date Month Year}
%\experiment{}
%Text
%
%----------------------------------------------------------------------------------------
%	IMAGENES
%----------------------------------------------------------------------------------------
% \begin{figure}[H] % Example of including images
% \begin{center}
% \includegraphics[width=0.5\linewidth]{example_figure}
% \end{center}
% \caption{Example figure.}
% \label{fig:example_figure}
% \end{figure}
% 
%----------------------------------------------------------------------------------------
%	TABLE
%---------------------------------------------------------------------------------------
% 
% \begin{table}[H]
% \begin{tabular}{l l l}
% \toprule
% \textbf{Groups} & \textbf{Treatment X} & \textbf{Treatment Y} \\
% \toprule
% 1 & 0.2 & 0.8\\
% 2 & 0.17 & 0.7\\
% 3 & 0.24 & 0.75\\
% 4 & 0.68 & 0.3\\
% \bottomrule
% \end{tabular}
% \caption{The effects of treatments X and Y on the four groups studied.}
% \label{tab:treatments_xy}
% \end{table}
% 
% Table \ref{tab:treatments_xy} shows that groups 1-3 reacted similarly to the two treatments but group 4 showed a reversed reaction.
% %----------------------------------------------------------------------------------------


%------------------------------------------------
%
%------------------------------------------------
\labday{Lunes, 1 Septiembre 2014}
\experiment{Sistema de gestion de versiones: Git}


\labday{Martes, 2 Septiembre 2014}
\experiment{Sensor LM35: Esquema de Conexion} %Semana 1 de Septiembre
\labday{Lunes, 13 Septiembre 2014}
\experiment{Selección de un microcontrolador}
Si bien la semana pasada se había optado por utilizar un microcontrolador de la familia $18Fxx$ 
resulta de interés el disminuir los costos del sistema por esta razón se contempla la posibilidad
de usar algunos de los siguientes $PICs$

\begin{table}[H]
    \begin{tabular}{l l l}
    \toprule
    \textbf{Patas} & \textbf{Antiguos} & \textbf{Nuevos} \\
    \toprule
    28 & 16F873 & 16F883\\
    28 & 16F876 & 16F886\\
    40 & 16F874 & 16F884\\
    40 & 16F877 & 16F887\\
    \bottomrule
    \end{tabular}
  \caption{Opciones de $PIC 16F$}
  \label{tab:comparacionPIC}
\end{table}

Dentro de la tabla\ref{tab:comparacionPIC} el $PIC16F884$ se encuentra disponible en el mercado local.
Esto lo posiciona por sobre los otros a la hora de la selección. Se comienza el análisis con este 
último PIC.

La información correspondiente al $PIC16F884$ se puede obtener de\\
\url{http://www.microchip.com/wwwproducts/Devices.aspx?dDocName=en026564}

\experiment{Seleccion de un entorno de desarrollo}
Buscamos encontrar un entorno de desarrollo que sea libre para poder ser usado sin restricciones.
Se plantean así las siguientes soluciones
\begin{itemize}
 \item Netbeans
 \item Eclipse
\end{itemize}
Sobre este ultimo se cuenta con experiencia en el desarrollo de aplicaciones embebidas y en C. Por tal
razon es que resulta mas interesante poder encontrar una solucion en este entorno.\\
El plugin mas difundido para poder realizar esta tarea es PIC C Builder:\\
\url{http://kr3l.wordpress.com/2008/11/02/using-eclipse-for-pic-development/}\\
\url{http://www.chiefdelphi.com/forums/showthread.php?t=35571}\\
\url{http://sourceforge.net/projects/piccbuilder/}\\
\url{https://github.com/ecdpalma/piccbuilder}\\
\url{http://marketplace.eclipse.org/content/pic-c-builder-eclipse#.VA3KIWDOqis}\\

Por otro lado, existe el IDE oficial de microchip llamado MPLAB X IDE:\\
\url{http://www.microchip.com/pagehandler/en_us/family/mplabx/}\\
el cual esta basado en NetBeans IDE from Oracle. Por ser ambos un proyecto
abierto resula muy interesante.\\
\url{http://microchip.wikidot.com/install:mplabx-lin64}




% \input{Sep/3.tex}
% \input{Sep/4.tex}
% 
% \labday{Lunes, 1 Septiembre 2014}
\experiment{Sistema de gestion de versiones: Git}


\labday{Martes, 2 Septiembre 2014}
\experiment{Sensor LM35: Esquema de Conexion} %Semana 1 de Oct
% \labday{Lunes, 13 Septiembre 2014}
\experiment{Selección de un microcontrolador}
Si bien la semana pasada se había optado por utilizar un microcontrolador de la familia $18Fxx$ 
resulta de interés el disminuir los costos del sistema por esta razón se contempla la posibilidad
de usar algunos de los siguientes $PICs$

\begin{table}[H]
    \begin{tabular}{l l l}
    \toprule
    \textbf{Patas} & \textbf{Antiguos} & \textbf{Nuevos} \\
    \toprule
    28 & 16F873 & 16F883\\
    28 & 16F876 & 16F886\\
    40 & 16F874 & 16F884\\
    40 & 16F877 & 16F887\\
    \bottomrule
    \end{tabular}
  \caption{Opciones de $PIC 16F$}
  \label{tab:comparacionPIC}
\end{table}

Dentro de la tabla\ref{tab:comparacionPIC} el $PIC16F884$ se encuentra disponible en el mercado local.
Esto lo posiciona por sobre los otros a la hora de la selección. Se comienza el análisis con este 
último PIC.

La información correspondiente al $PIC16F884$ se puede obtener de\\
\url{http://www.microchip.com/wwwproducts/Devices.aspx?dDocName=en026564}

\experiment{Seleccion de un entorno de desarrollo}
Buscamos encontrar un entorno de desarrollo que sea libre para poder ser usado sin restricciones.
Se plantean así las siguientes soluciones
\begin{itemize}
 \item Netbeans
 \item Eclipse
\end{itemize}
Sobre este ultimo se cuenta con experiencia en el desarrollo de aplicaciones embebidas y en C. Por tal
razon es que resulta mas interesante poder encontrar una solucion en este entorno.\\
El plugin mas difundido para poder realizar esta tarea es PIC C Builder:\\
\url{http://kr3l.wordpress.com/2008/11/02/using-eclipse-for-pic-development/}\\
\url{http://www.chiefdelphi.com/forums/showthread.php?t=35571}\\
\url{http://sourceforge.net/projects/piccbuilder/}\\
\url{https://github.com/ecdpalma/piccbuilder}\\
\url{http://marketplace.eclipse.org/content/pic-c-builder-eclipse#.VA3KIWDOqis}\\

Por otro lado, existe el IDE oficial de microchip llamado MPLAB X IDE:\\
\url{http://www.microchip.com/pagehandler/en_us/family/mplabx/}\\
el cual esta basado en NetBeans IDE from Oracle. Por ser ambos un proyecto
abierto resula muy interesante.\\
\url{http://microchip.wikidot.com/install:mplabx-lin64}




% \input{Oct/3.tex}
% \input{Oct/4.tex}
% 
% \labday{Lunes, 1 Septiembre 2014}
\experiment{Sistema de gestion de versiones: Git}


\labday{Martes, 2 Septiembre 2014}
\experiment{Sensor LM35: Esquema de Conexion} %Semana 1 de Nov
% \labday{Lunes, 13 Septiembre 2014}
\experiment{Selección de un microcontrolador}
Si bien la semana pasada se había optado por utilizar un microcontrolador de la familia $18Fxx$ 
resulta de interés el disminuir los costos del sistema por esta razón se contempla la posibilidad
de usar algunos de los siguientes $PICs$

\begin{table}[H]
    \begin{tabular}{l l l}
    \toprule
    \textbf{Patas} & \textbf{Antiguos} & \textbf{Nuevos} \\
    \toprule
    28 & 16F873 & 16F883\\
    28 & 16F876 & 16F886\\
    40 & 16F874 & 16F884\\
    40 & 16F877 & 16F887\\
    \bottomrule
    \end{tabular}
  \caption{Opciones de $PIC 16F$}
  \label{tab:comparacionPIC}
\end{table}

Dentro de la tabla\ref{tab:comparacionPIC} el $PIC16F884$ se encuentra disponible en el mercado local.
Esto lo posiciona por sobre los otros a la hora de la selección. Se comienza el análisis con este 
último PIC.

La información correspondiente al $PIC16F884$ se puede obtener de\\
\url{http://www.microchip.com/wwwproducts/Devices.aspx?dDocName=en026564}

\experiment{Seleccion de un entorno de desarrollo}
Buscamos encontrar un entorno de desarrollo que sea libre para poder ser usado sin restricciones.
Se plantean así las siguientes soluciones
\begin{itemize}
 \item Netbeans
 \item Eclipse
\end{itemize}
Sobre este ultimo se cuenta con experiencia en el desarrollo de aplicaciones embebidas y en C. Por tal
razon es que resulta mas interesante poder encontrar una solucion en este entorno.\\
El plugin mas difundido para poder realizar esta tarea es PIC C Builder:\\
\url{http://kr3l.wordpress.com/2008/11/02/using-eclipse-for-pic-development/}\\
\url{http://www.chiefdelphi.com/forums/showthread.php?t=35571}\\
\url{http://sourceforge.net/projects/piccbuilder/}\\
\url{https://github.com/ecdpalma/piccbuilder}\\
\url{http://marketplace.eclipse.org/content/pic-c-builder-eclipse#.VA3KIWDOqis}\\

Por otro lado, existe el IDE oficial de microchip llamado MPLAB X IDE:\\
\url{http://www.microchip.com/pagehandler/en_us/family/mplabx/}\\
el cual esta basado en NetBeans IDE from Oracle. Por ser ambos un proyecto
abierto resula muy interesante.\\
\url{http://microchip.wikidot.com/install:mplabx-lin64}




% \input{Nov/3.tex}
% \input{Nov/4.tex}

%----------------------------------------------------------------------------------------
%	FORMULAE AND MEDIA RECIPES
%----------------------------------------------------------------------------------------

\labday{} % We don't want a date here so we make the labday blank
\begin{center}
\HRule \\[0.4cm]
{\huge \textbf{Notas}}\\[0.4cm] % Heading
\HRule \\[1.5cm]
\end{center}

Precios a tener en cuenta:
\begin{itemize}
 \item Bateria de Gel: Disponibles en Electronica Mendoza\\
 Precio: $\$270$\\
 Se plantea la necesidad de utilizar baterias en caso emergencia?
 \item Placas de Fibra para PCB:\\
 Precio: $20x20=\$75$ y $15x15=\$39$
\end{itemize}

\subsection{SMS}
La empresa CEM(Ver: \url{http://www.cemsrl.com.ar/}) cuenta con un modulo GSM para 
alarmas domiciliarias G100: \url{http://www.cemsrl.com.ar/productos/productos.php?marca=11&producto=223}\\
Resulta interesante poder adaptar este modulo para ser utilizado mediante la beaglebone.


\subsection{Dudas}
\begin{itemize}
 \item Class-B Safety Software: Resulta de interés implementarlo??: 
 \url{http://www.microchip.com/pagehandler/en-us/technology/homeAppliance/classbsafetysoftware.html}
\end{itemize}

\subsection{Corriente}
Para poder tener una lectura de la corriente se puede utilizar el sensor
$LTS~6-NP$ de LEM. Consultar disponibilidad en el mercado local.loca
%----------------------------------------------------------------------------------------

\end{document}